\documentclass[11pt]{article}
\usepackage{amsmath}
\usepackage{amssymb}
\usepackage[margin=.5in,left=.5in]{geometry} 
\usepackage{amsthm}
\setlength{\parindent}{0pt}

\begin{document}

\begin{titlepage}
    \centering
    \vspace*{\fill} % Push content to vertical center
    {\Large Ordinary Differential Equations} \\[0.5cm]
    {\Large Chapter 1.2 Notes: Solutions \& Inital Value Problems} \\[0.5cm]
    {\large Victor C. Van} \\[1cm]
    {\today}
    \vspace*{\fill} % Push content to vertical center
\end{titlepage}

An \textit{n}th order ordinary differential equation is an equality involving the independent variable x,  the dependent variable y, and the first \textit{n} derivative of y.
Examples are:

\[
    \text{Second Order}: x^{2} + \frac{d^{2}y}{dx^{2}} + y = x^{3}
\]

\[
    \text{Second Order}: \sqrt{1- (\frac{d^{2}y}{dx^{2}})^{2}} - y = 0
\]

\[
    \text{Fourth Order}: \frac{d^{4}y}{dx^{4}} = xy
\]

Thus a general form for an nth order equation would be:

\[
    F(x,y, \frac{dy}{dx}, \cdots, \frac{d^{n}y}{dx^{n}} = 0) \tag{1}
\]

where \textit{F} is a function of the independent variable x, the dependent variable y, and the derivative of y up to order n; that is $x,y, \cdots, \frac{d^{n-1}y}{dx^{n-1}}$.
We assume thst the equation holds for all x in an interval \textit{I} (which masy or may not include its endpoints:
$a \leq x \leq b, a < x \leq b, etc).$ In many cases, we can isolate the highest-order term $\frac{d^{n}y}{dx^{n}}$, and write (1) as:

\[
    \frac{d^{n}y}{dx^{n}} = f(x,y, \frac{dy}{dx}, \cdots, \frac{d^{n}y}{dx^{n}}) \tag{2}
\]

which is often preferable to (1) for theoretical and computational purposes.


\fbox{
\begin{minipage}{0.95\linewidth} % Adjust the width as needed
\textbf{EXPLICIT SOLUTION}\\
\textbf{Definition 1.} A function $\phi(x)$ that when substituted for $y$ in equation (1) [or (2)] satisfies the equation for all $x$ in the interval \textit{I} is called an \textbf{explicit solution} to the equation on \textit{I}.
\end{minipage}}

\textbf{EXAMPLE 1:} Show that $\phi(x) = x^{2} - x^{-1}$ is an explicit solution today

\[
    \frac{d^{2}y}{dx^{2}} - \frac{2}{x^{2}}y = 0.
\]

\textbf{Solution}: The functions $\phi(x) = x^{2} - x^{-1}, \phi'(x) = 2x + x^{-2}$, and $\phi''(x) = 2 - 2x^{-3}$ are defined for all
$x \not = 0$. Substitution of $\phi(x)$ for $y$ in the equation (3) gives:

\[
    (2-2x^{-3})- \frac{2}{x^{2}}(x^{2}-x^{-1}) =  (2-2x^{-3}) - (2-2x^{-3}) = 0.
\]

\textbf{EXAMPLE 2:} Show that for \textit{any} choice of the constants $c_1$ and $c_2$, the function

\[
    \phi(x) = c_{1}e^{-1} + c_2e^{2x}
\]

is an explicit solution to

\[
    y'' -  y' -2y = 0. \tag{4}
\]


\newpage
\[
    a \land b, c \lor d, \exists
\]
\end{document}