\documentclass{article}
\usepackage{graphicx} % Required for inserting images
\usepackage{amssymb}
\usepackage{gensymb}
\usepackage{esdiff}

\title{MAT251 - Calculus 1 Project}
\author{Victor Van}
\date{December 4th 2023}

\begin{document}

\maketitle

\newpage

\setcounter{tocdepth}{1} % Set table of contents depth to include sections only

\tableofcontents

\section*{Problem 1 (25 points) - Page 3}
The distance ($d$) at which the ball is thrown depends on the initial velocity ($v$), the height of release ($h$), and the angle ($\theta$) when the ball is released.

One model developed for this situation is:
\begin{equation}
    d = \frac{v^2 \sin(\theta) \cos(\theta) + v\cos(\theta) \sqrt{(vsin(\theta))^2 + 64h}}{32}
\end{equation}

\section*{Problem 2 (12 points) - Pages 4-5}
It is common to hear the sound of a flying airplane and look in the wrong place in the sky to see the plane. Suppose that a plane is traveling horizontally in your direction at a speed of 200 mph at an altitude of 3000 ft. You hear the sound at what seems to be an angle of elevation of 20°.

\section*{Problem 3 (6 points) - Page 6}
Metals expand when heated and contract when cooled. The dimensions of a piece of laboratory equipment are sometimes so critical that the shop where the equipment is made must be held at the same temperature as the laboratory where the equipment is to be used. A typical aluminum bar that is 10 cm wide at 70°F will be 
\begin{equation}
    w = 10 + (t-70) \times 10^{-4}
\end{equation}
centimeters wide at a nearby temperature \(t\). Suppose that you are using a bar like this in a gravity wave detector, where its width must stay within 0.0005 cm of the ideal 10 cm. How close to \(t_0 = 70°F\) must you maintain the temperature to ensure that this tolerance is not exceeded?

\section*{Problem 4 (18 points) - Page 7}
The average price, in dollars, of a ticket for a Major League baseball game \(t\) years after 2000 can be estimated by
\begin{equation}
    p(t) = 15.50 - 0.19t + 0.09t^2
\end{equation}

\section*{Problem 5 (24 points) - Pages 8-9}
A 10-foot ladder is propped up against a wall so that the base of the ladder is 6 feet away from the wall. Suppose the ladder starts sliding down the wall at a rate of 2 feet per second. Give the exact value and approximate value (2 dec) for each.

\newpage

\section*{Problem 1:}
1a) To see how the angle affects distance, let v = 44 ft/sec and h = 6.25 ft.
Calculate distance d, to the nearest thousandth, for $\theta$ = 40°, 42.5°, 45°. List all distances.
\newline \newline \textbf{Givens:}
\newline
v = 44ft/sec
\newline
h = 6.25ft
\newline \newline
Let d = f(v, h, $\theta$)
\begin{equation}
    f(v, h, \theta) = \frac{v^2 \sin(\theta) \cos(\theta) + v\cos(\theta) \sqrt{(vsin(\theta))^2 + 64h}}{32}
\end{equation}
\subsection*{1a)}
f(44, 6.25, 40°) $\approx$ 66.277ft
\newline
f(44, 6.25, 42.5°) $\approx$ 66.456ft
\newline
f(44, 6.25, 45°) $\approx$ 66.211ft
\subsection*{1b)}
Distance does seem to be affected by $\theta$, but are relatively negligible However, there's after $\theta$ reaches a certain threshold, distance decreases.
\subsection*{1c)}
f(35, 6.25, 42.5°) $\approx$ 44.042ft
\newline
f(40, 6.25, 42.5°) $\approx$ 55.889ft
\newline
f(45, 6.25, 42.5°) $\approx$ 69.250ft
\subsection*{1d)} Velocity affects distance much more compared to $\theta$. The ideal velocity for maximum distance is in between 35 and 45.
\subsection*{1e)}
Distance is affected by velocity significantly more than $\theta$. However, $\theta$ still affects distance. To improve performance, a shotputter should focus on their power which affects the velocity of the ball. Secondly, they should focus on the angle that they throw the ball - as it still affects distance albeit more minorly. The angle influences the distance that the shotput ball is thrown; the ideal angle for the furthest throw is between 40° and 45°. The velocity affects the distance the shotput ball is thrown to a much greater degree, ideally in between 35 and 45 feet per second.

\newpage

\section*{Problem 2:}

\subsection*{2a)}

\begin{figure}[ht]
\centering
\includegraphics[scale=.5]{calculus1projectnumber2pic1.png}
\end{figure}

\begin{figure}[ht]
\centering
\includegraphics[scale=.5]{calculus1projectnumber2pic2.png}
\end{figure}

\begin{figure}[ht]
    \centering
    \includegraphics[width=0.5\linewidth]{calculus1projectnumber2pic3.png}
\end{figure}

Givens:
\newline speed of sound: s = 1,125ft/s
\newline y = 3,000ft
\newline $\theta$ = 20°
\newline V$_{p}$ = 200mph
\newline $\Delta$x$_{p}$ = change of position of plane
\newline Find $\beta$

\subsection*{2b)}

\begin{equation}
    x_{s} = x_{p} + \Delta x_{p}
\end{equation}

\begin{equation}
    tan(\theta) = \frac{y}{x_{s}}
\end{equation}

\begin{equation}
    sin(\theta) = \frac{y}{z_{s}}
\end{equation}

\begin{equation}
    x_{s} = ycot{\theta}
\end{equation}

\begin{equation}
    z_{s} = st \therefore t = z_{s}s^{-1} \therefore t = s^{-1}ycsc(\theta)
\end{equation}

\begin{equation}
    \Delta x_{p} = v_{p}t
\end{equation}

\begin{equation}
    x_{p} = x_{s} - \Delta x_{p}
\end{equation}

$\therefore$

\begin{equation}
    x_{p} = ycot(\theta) - v_{p}s^{-1}ycsc(\theta) \therefore x_{p} = ycot(\theta) - \frac{176}{675}ycsc(\theta)
\end{equation}

\begin{equation}
    tan(\beta) = \frac{y}{x_{p}} \therefore tan(\beta) = \frac{1}{cot(\theta)-\frac{176}{675}csc(\theta)} \therefore \beta = tan^{-1} (\frac{1}{cot(\theta)-\frac{176}{675}csc(\theta)})
\end{equation}
\begin{equation}
    \beta \approx 26.74\degree    
\end{equation}

\newpage

\section*{Problem 3:}
\subsection*{3a)} How close to t$_0$ = 70°F must you maintain to ensure that this tolerance is not exceeded?
\newline \newline
Givens:
\newline aluminum bar: 10cm at 70°F
\newline
w = width
\newline
t = temperature
\newline
width must stay 0.0005cm of the ideal 10cm
\begin{equation}
    w = 10+(t-70)(10^{-4})
\end{equation}

w$\pm$a $\rightarrow$ 10-a$\leq$w$\leq$10+a
\begin{equation}
    10-a\leq 10+(t-70)(10^{-4})\leq10+a
\end{equation}
t = t$_0 \pm $5°F

\newpage

\section*{Problem 4:}
Givens:
\newline
t = years after 2000
\subsection*{4a)} Find the rate of change of the average ticket price with respect to the year.
\begin{equation}
    p(t) = 15.50-0.19t+0.09t^2
\end{equation}
\newline Rate of change of average price tickets:
\begin{equation}
    p'(t) = 0.18t-0.19
\end{equation}

\subsection*{4b)} Average price of ticket in 2022:
\begin{equation}
    p(22) = \textdollar54.88
\end{equation}

\subsection*{4c)} Rate of change of average price ticket in 2022:
\begin{equation}
    p'(22) = \textdollar3.77
\end{equation}

\newpage

\section*{Problem 5:}

\subsection*{5a)}

\begin{figure}[ht]
\centering
\includegraphics[scale=.5]{calculus1projectnumber5.png}
\end{figure}

Givens:

x = 6 ft, z = 10 ft $\therefore$ y = 8 ft by using Pythagorean theorem

$\diff{y}{t}$ = -2 ft/sec

\subsection*{5b)}
Find $\diff{x}{t}$

Take the derivative of x$^2$+y$^2$ = z$^2$ in respect to time and isolate $\diff{x}{t}$
\begin{equation}
    \diff{x}{t} = \frac{-2y(\diff{y}{t}) + 2z(\diff{z}{t})}{2x}
%    {-2y(\diff{y}{t}+2z(\diff{z}}{t}/2x)
\end{equation}

Plug in the variables and simplify:

\begin{equation}
    \diff{x}{t} = \frac{8}{3} ft/sec \approx 2.67 ft/sec
\end{equation}

\subsection*{5c)}
To find the angle of the ladder makes with the wall is changing, let the angle opposite of $\theta$ $\beta$
\newline Find $\diff{\beta}{t}$

\begin{equation}
    sin(\beta) = \frac{x}{z}
\end{equation}

Take the derivative in respect to 5t.

\begin{equation}
    cos(\beta) \diff{\beta}{t} = (\frac{1}{z})(\frac{dx}{dt})
\end{equation}

Isolate $\diff{\beta}{t}$

\begin{equation}
   \diff{\beta}{t} =  (\frac{1}{z})(sec(sin^{-1}(\frac{x}{z}))(\diff{x}{t})
\end{equation}

If the ladder is sliding down the wall $\beta$ is increasing, and $\theta$ is decreasing. The calculations here shows that $\beta$ is negative. Therefore, find $|\diff{\beta}{t}|$

\begin{equation}
    |\diff{\beta}{t}| = \frac{1}{4} rad/sec \approx 0.25 rad/sec
\end{equation}

Out of curiosity, we can also find $\diff{\theta}{t}$.

Take cos($\theta$)

\begin{equation}
    cos(\theta) = \frac{x}{z}
\end{equation}

Take the derivative in respective to t. Note that z is a constant because the length of the ladder doesn't change.

\begin{equation}
    -sin(\theta)(\diff{\theta}{t}) = (\frac{1}{z})(\diff{x}{t})
\end{equation}

Plug in variables and isolate $\diff{\theta}{t}$

\begin{equation}
\diff{\theta}{t} = - \frac{1}{3} rad/sec \approx -0.33 rad/sec    
\end{equation}

\subsection*{5d)}
Find $\diff{a}{t}$

Consider the formula of the area of a triangle
\begin{equation}
    a = \frac{1}{2}xy
\end{equation}

Take the derivative of it to find the rate of change of area in respect to time.

\begin{equation}
    \diff{a}{t} = \frac{1}{2}(y \diff{x}{t} + x \diff{y}{t})
\end{equation}

Plug in variables, simplify, and isolate $\diff{a}{t}$

\begin{equation}
    \diff{a}{t} = \frac{14}{3} ft/s^2 \approx 4.67 ft/s^2
\end{equation}

\end{document}